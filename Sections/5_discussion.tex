\section{Conclusion}
\label{sec:discussion}

FaceLift is a  framework that automatically beautifies urban scenes by combining recent approaches of Generative Adversarial Networks and Deep Convolutional Networks. To make it usable by practitioners, the framework is also able to explain which urban elements have been added/removed during the beautification process. 

There are still two main limitations though. One is data bias. The framework is as good as its training data, and more work has to go into collecting reliable ground truth data on human perceptions. This data should ideally be stratified according to the people's characteristics that  impact their perceptions. The other main limitation is that generative models are hard to control, and more work has to go into offering principled ways of fine-tuning the generative process.

Despite these limitations, FaceLift has the potential to support urban interventions  in scalable  and replicable ways: it can be applied to an entire city (scalable), across a variety of cities (replicable). To turn existing spaces into something more beautiful, that will still be the duty of architecture. Yet, with technologies similar to FaceLift more readily available, the complex job of recreating restorative spaces in an increasingly urbanized world will be greatly simplified.  


%After all, ``we delight in complexity to which genius have lent an appearance of simplicity.''~\cite{de2008architecture} In the context of future work, that genius is represented by future technologies that will help us deal with the complexity of our cities.



%\subsection{Limitations and biases}
%Like any supervised deep learning based framework, this work is only able to learn what is present in the data. Hence the method of acquiring annotations  for urban images can introduce huge biases in the model. The current model is trained on images acquired from the study on streetscore \cite{naik2014streetscore}. However their annotation is open to general public and there is not way we can remove biases that come with culture and location, in a highly subjective effect like beauty. Moreover because the pair wise choice is simply done by clicking one of the two images, the data might have noise introduced by non-serious participants. Such biases are bound to be picked up by the deep learning model. One can argue that the preference of our model for greenery , is a form of bias in the data. Another bias introduced because of data is the model's lack of preference to pedestrians. This bias was established well in advance because Google tries to remove most of the people from their street view images for privacy reasons. Hence people, which make up a major aspect of urban vitality, are completely missing from most dataset images and hence from the facelift transformations. 
%Another Limitation of our work is in the metric formation. The computational metrics developed to capture the real urban design metrics are designed using heuristics. There needs to be more crowd and expert validation to establish the validity of their formulation. 
%
%\subsection{Future work}
%The framework is generalizable for geotagged and annotated images. The aim of this paper is to propose a framework with uses state of art methods in generative models to understand perception of emotions in urban images and explain them. As an extension, understanding how intervention would look like against outcome variables such as depression, safety or mental well-being in general would be very valuable.