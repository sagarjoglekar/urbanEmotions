%%%%%%%%%%%%%%%%%%%%%%%%%%%%%%%%%%%%%%%%%%%%%%%%%%%%%%%%%%%%%%%%
%% RSOS First rebuttal
%%%%%%%%%%%%%%%%%%%%%%%%%%%%%%%%%%%%%%%%%%%%%%%%%%%%%%%%%%%%%%%%

\section{Discussion}
FaceLift is a  framework that automatically beautifies urban scenes by combining recent approaches of Generative Adversarial Networks and Deep Convolutional Networks. To make it usable by practitioners, the framework is also able to explain which urban elements have been added/removed during the beautification process. 

\subsection{Limitations}

\la{Facelift still faces some important challenges. The main limitation is that generative image models are still hard to control, especially when dealing with complex scenes containing multiple elements. Some of the beautifications suggested by our tool modify the scenes too dramatically to use them as blueprints for urban interventions (e.g., shifting buildings or broadening roads). This undesired effect is compounded by the restricted size and potential biases of the data that we use both for training and for selecting the scene most similar to the machine-generated image---which might result, for example, in generated scenes that are set in seasons or weather conditions that differ from the input image. To address these limitations, more work has to go into offering principled ways of fine-tuning the generative process, as well as into collecting reliable ground truth data on human perceptions. This data should ideally be stratified according to the people's characteristics that impact their perceptions. Performance assessment frameworks for the built environment (like the Living Building Challenge\footnote{\url{https://living-future.org/lbc/beauty-petal}}) could provide a good source of non-traditional qualitative measures useful for training and validating the Facelift algorithm.}
\la{Facelift still faces some important challenges. The main limitation is that generative image models are still hard to control, especially when dealing with complex scenes containing multiple elements. Some of the beautifications suggested by our tool modify the scenes too dramatically to use them as blueprints for urban interventions (e.g., shifting buildings or broadening roads). This undesired effect is compounded by the restricted size and potential biases of the data that we use both for training and for selecting the scene most similar to the machine-generated image---which might result, for example, in generated scenes that are set in seasons or weather conditions that differ from the input image. To address these limitations, more work has to go into offering principled ways of fine-tuning the generative process, as well as into collecting reliable ground truth data on human perceptions. This data should ideally be stratified according to the people's characteristics that impact their perceptions. Performance assessment frameworks for the built environment (like the Living Building Challenge\footnote{\url{https://living-future.org/lbc/beauty-petal}}) could provide a good source of non-traditional qualitative measures useful for training and validating the Facelift algorithm.}

\la{Another important limitation has to do with the complexity of the notion of beauty. There exists a wide spectrum of perceptive measures by which urban scenes could be considered beautiful. This is because the ``essence'' of a place is socio-cultural and time-specific~\cite{norberg80genius}. The collective perception of the urban environment evolves over time as its appearance and function change~\cite{brand1995buildings} as a result of shifting cultures, new urban policies, and placemaking initiatives~\cite{foth2017lessons}. An undiscerning, mechanistic application of machine learning tools to urban beautification might be undesirable because current technology cannot take into account most of these crucial aspects. FaceLift is no exception, and this is why we envision its use as a way to support new forms of placemaking rather than as a tool to replace traditional approaches. Nevertheless, we emphasize the need of a critical reflection on the implications of deploying such a technology, even when just in support of placemaking activities. In particular, it would be beneficial to study the impact of the transformative effect of FaceLift-inspired interventions on the ecosystem of the city~\cite{dourish2016algorithms,kitchin2017thinking} as well as exploring the need to pair its usage with practices and principles that might reduce any potential undesired side effect~\cite{kitchin2016ethics}}

%One of the biggest is the data bias. The framework is as good as its training data, and more work has to go into collecting reliable ground truth data on human perceptions. This data should ideally be stratified according to the people's characteristics that  impact their perceptions.  Finally the other main limitation is that generative models are hard to control, and more work has to go into offering principled ways of fine-tuning the generative process.

%Another important limitation is the possibility that our metrics are not actually capturing urban metrics like ``Walkability'' and ``Openness''. It is possible that the PlacesNet~\cite{zhou2014learning} labels could actually be capturing something akin to walkability. We try to reduce this risk by using definitions found in the literature~\cite{quercia15thedigital,ewing2013measuring} as a guidance to classify the PlacesNet labels into the 4 categories.


\subsection{Conclusion}
Despite these limitations, FaceLift has the potential to support urban interventions  in scalable  and replicable ways: it can be applied to an entire city (scalable), across a variety of cities (replicable). 

\la{We conceived FaceLift not as a technology to \emph{replace} the decision making process of planners and architects, but rather as a tool to \emph{support} their work. FaceLift could aid the creative process of beautification of a city by suggesting imagined versions of what urban spaces could become after applying certain sets of interventions. We do not expect machine-generated scenes to equal the quality of designs done by experts. However, unlike the work of an expert, FaceLift is able to generate beautified scenes  very fast (in seconds) and at scale (for an entire city), while quickly providing a numerical estimate of how much some urban elements should change to increase beauty. The user study we conducted suggests that these features make it possible to inspire the work of decision makers and to nudge then into considering alternative approaches to urban interventions that might not otherwise be apparent. We believe this source of inspiration could advantage non-experts too, for example by helping residents to imagine a possible future for their cities and by motivating citizen action in the deployment of micro-interventions.}

To turn existing spaces into something more beautiful, that will still be the duty of architecture. Yet, with technologies similar to FaceLift more readily available, the complex job of recreating restorative spaces in an increasingly urbanized world will be greatly simplified.  

%%%%%%%%%%%%%%%%%%%%%%%%%%%%%%%%%%%%%%%%%%%%%%%%%%%%%%%%%%%%%%%%
 %% RSOS V2
%%%%%%%%%%%%%%%%%%%%%%%%%%%%%%%%%%%%%%%%%%%%%%%%%%%%%%%%%%%%%%%%
%\section{Conclusion}
%\label{sec:discussion}
%
%FaceLift is a  framework that automatically beautifies urban scenes by combining recent approaches of Generative Adversarial Networks and Deep Convolutional Networks. To make it usable by practitioners, the framework is also able to explain which urban elements have been added/removed during the beautification process. 
%
%There are still two main limitations though. One is data bias. The framework is as good as its training data, and more work has to go into collecting reliable ground truth data on human perceptions. This data should ideally be stratified according to the people's characteristics that  impact their perceptions. The other main limitation is that generative models are hard to control, and more work has to go into offering principled ways of fine-tuning the generative process.
%
%Despite these limitations, FaceLift has the potential to support urban interventions  in scalable  and replicable ways: it can be applied to an entire city (scalable), across a variety of cities (replicable). To turn existing spaces into something more beautiful, that will still be the duty of architecture. Yet, with technologies similar to FaceLift more readily available, the complex job of recreating restorative spaces in an increasingly urbanized world will be greatly simplified.  

%%%%%%%%%%%%%%%%%%%%%%%%%%%%%%%%%%%%%%%%%%%%%%%%%%%%%%%%%%%%%%%%
%After all, ``we delight in complexity to which genius have lent an appearance of simplicity.''~\cite{de2008architecture} In the context of future work, that genius is represented by future technologies that will help us deal with the complexity of our cities.



%\subsection{Limitations and biases}
%Like any supervised deep learning based framework, this work is only able to learn what is present in the data. Hence the method of acquiring annotations  for urban images can introduce huge biases in the model. The current model is trained on images acquired from the study on streetscore \cite{naik2014streetscore}. However their annotation is open to general public and there is not way we can remove biases that come with culture and location, in a highly subjective effect like beauty. Moreover because the pair wise choice is simply done by clicking one of the two images, the data might have noise introduced by non-serious participants. Such biases are bound to be picked up by the deep learning model. One can argue that the preference of our model for greenery , is a form of bias in the data. Another bias introduced because of data is the model's lack of preference to pedestrians. This bias was established well in advance because Google tries to remove most of the people from their street view images for privacy reasons. Hence people, which make up a major aspect of urban vitality, are completely missing from most dataset images and hence from the facelift transformations. 
%Another Limitation of our work is in the metric formation. The computational metrics developed to capture the real urban design metrics are designed using heuristics. There needs to be more crowd and expert validation to establish the validity of their formulation. 
%
%\subsection{Future work}
%The framework is generalizable for geotagged and annotated images. The aim of this paper is to propose a framework with uses state of art methods in generative models to understand perception of emotions in urban images and explain them. As an extension, understanding how intervention would look like against outcome variables such as depression, safety or mental well-being in general would be very valuable.