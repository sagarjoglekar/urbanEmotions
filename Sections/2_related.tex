\section{Related Work}
\label{sec:related}
%\ns{This needs more work.}
Previous work has focused on: collecting ground truth data about how people perceive urban spaces; predicting urban qualities from visual data; and generating synthetic images that enhance a given quality (e.g., beauty). 

%\mbox{}\\
\vspace{4pt}\noindent
\textbf{Perception of physical spaces.} From Jane Jacobs's seminal work on urban vitality~\cite{jacobs1961death} to Christopher Alexander's cataloging of typical ``patterns'' of good urban design~\cite{alexander1977pattern}, there has been a continuous effort to understand what makes our cities livable and enjoyable.  In the fields of psychology, environmental design and behavioral sciences, research has studied the relationship between urban aesthetics~\cite{real2000classification} and a variety of objective measures  (e.g.,  scene complexity~\cite{kaplan1972rated}, presence of nature~\cite{kaplan1989experience}) and subjective ones (e.g., people's affective responses~\cite{ulrich1983aesthetic}).  

%All this work showed that human perceptions of urban spaces are predictable, and that had made it possible to study cities at scale with modern statistical tools. 
%\mbox{}\\

%\mbox{}\\
\vspace{4pt}\noindent
\textbf{Ground truth of urban perceptions.} So far, the most detailed studies of perceptions of urban environments and their visual appearance have relied on personal interviews and  observation: some researchers relied on annotations of video recordings by experts~\cite{sampson04seeing}, while others have used participant ratings of simulated (rather than existing) street scenes~\cite{lindal2012}. The Web has recently been used to survey a large number of individuals. Place Pulse is a website that asks a series of binary perception questions (such as `Which place looks safer [between the two]?') across a large number of geo-tagged images~\cite{salesses2013collaborative}. In a similar way, Quercia \emph{et al.} collected pairwise judgments about the extent to which urban scenes are considered quiet, beautiful and happy~\cite{quercia2014aesthetic} to then recommend pleasant paths in the city~\cite{quercia2014shortest}. They were then able to analyze the scenes together with their ratings using image-processing tools, and found that the amount of greenery in any given scene was associated with all three attributes and that cars and fortress-like buildings were associated with sadness. Taken all together, their results pointed in the same direction: urban elements that hinder social interactions were undesirable, while elements that increase interactions were the ones that should be integrated by urban planners to retrofit cities for happiness. Urban perceptions translate in concrete outcomes. Based on 3.3k self-reported survey responses,  Ball et al.~\cite{ball2001perceived} found that urban scenes that are aesthetically beautiful not only are visually  pleasurable but also promote walkability. Similar findings were obtained by Giles et al. \cite{giles2005increasing}.
%\mbox{}\\

\vspace{4pt}\noindent
\textbf{Deep learning and the city.} Computer vision techniques have increasingly become more sophisticated. Deep learning techniques, in particular, have been recently used to accurately predict urban beauty~\cite{dubey2016deep,seresinhe2017using}, urban change~\cite{naik2017computer}, and even crime~\cite{DeNadai16,arietta2014city}.  Recent works have also showed the utility of deep learning techniques in predicting house prices from urban frontages~\cite{frontage}, and from a combination of satellite data and street view images~\cite{law2018take}.
%\mbox{}\\

%\mbox{}\\
\vspace{4pt}\noindent
\textbf{Generative models.} Since the introduction of Generative Adversarial Networks (GANs)~\cite{goodfellow2014generative}, deep learning has been used not only to analyze existing images but also to generate new ones altogether. This family of deep networks has evolved into various forms, from super resolution image generators~\cite{ledig2017photo} to fine-grained in-painting technologies~\cite{pathak2016context}. Recent approaches have been used to generate images conditioned on specific visual attributes~\cite{yan2015attribute2image}, and these images range from faces~\cite{taigman2016unsupervised} to people~\cite{ma2018disentangled}. In a similar vein, Nguyen \emph{et al.}~\cite{nguyen2016synthesizing} used generative networks to create a natural-looking image that maximizes a specific neuron (the beauty neuron).  In theory, the resulting image is the one that ``best activates'' the neuron under consideration. In practice, it is still a synthetic template that needs further processing to look realistic.   Finally, with the recent advancement in Augmented Reality, the application of GANs to generate urban objects in simulated urban scenes have also been attempted~\cite{alhaija2018augmented}. 
%\mbox{}\\

%\mbox{}
\vspace{4pt}
To sum up, a lot of work has gone into collecting ground truth data about how people tend to perceive urban spaces, and into building accurate predictions models of urban qualities. Yet little work has gone into models that generate realistic urban scenes that maximize a specific property and that offer human-interpretable explanations of what they generate. 


