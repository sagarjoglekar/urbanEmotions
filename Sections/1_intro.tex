\section{Introduction}


Whether a street is considered beautiful is subjective, yet research has shown that there are specific urban elements that are universally considered beautiful: from greenery, to small streets, to memorable spaces~\cite{alexander1977pattern, quercia2014aesthetic,salesses2013collaborative}. These elements are those that contribute to the creation of what the urban sociologist Jane Jacobs called `urban vitality'~\cite{jacobs1961death}. 


Given that, it comes as no surprise that computer vision techniques can automatically analyse pictures of urban scenes and accurately determine the extent to which these scenes are considered, \emph{on average}, beautiful.  Deep learning has greatly contributed to increase these techniques' accuracy~\cite{dubey2016deep}.

However, urban planners and architects are interested in urban interventions and, as such, they
would welcome machine learning technologies that help them recreate beauty in urban design~\cite{de2008architecture} rather than simply predicting beauty scores. As we shall see in Section~\ref{sec:related}, deep learning, by itself, is not fit for purpose. It is not meant to recreate beautiful scenes, not least because it cannot provide any explanation on why a scene is deemed beautiful, or which urban elements are predictors of beauty.


To partly fix that, we propose a deep learning framework (which we name  FaceLift) that is able to both \emph{generate} a beautiful scene (or, better, \emph{beautify} an existing one) and \emph{explain} which parts make that scene beautiful. \la{Our work contributes to the field of urban informatics, an interdisciplinary area of research that studies practices and experiences across urban contexts and creates new digital tools to improve those experiences~\cite{foth2009handbook,foth2011urban}}. Specifically, we make two main contributions:

\begin{itemize}
\item We propose a deep learning framework that is able to learn whether a particular set of Google Street Views (urban scenes) are beautiful or not, and based on that training, the framework is then able to both \emph{beautify} existing views and \emph{explain} which urban elements  make them beautiful (Section~\ref{sec:framework}). 
%These explanations are automatically extracted with computer vision tools. 

\item We quantitatively evaluate whether the framework is able to actually produce beautified scenes (Section~\ref{sec:evaluation}). We do so by proposing a family of five urban design metrics that we have formulated based on a thorough review of the literature in urban planning. For all these five metrics, the framework passes with flying colours: with minimal interventions, beautified scenes are twice as walkable as the original ones, for example. Also, after building an interactive tool with ``FaceLifted'' scenes in Boston and presenting it to twenty experts in architecture,  we found that the majority of them agreed on three main areas of our work's impact: decision making, participatory urbanism, and the promotion of restorative spaces. 
\end{itemize}





