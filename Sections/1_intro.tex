\section{Introduction}


Whether a street is considered beautiful is subjective, yet research has shown that there are specific urban elements that are universally considered beautiful: from greenery, to small streets, to memorable spaces~\cite{alexander1977pattern, quercia2014aesthetic,salesses2013collaborative}. These elements are those that contribute to the creation of what the urban sociologist Jane Jacobs called `urban vitality'~\cite{jacobs1961death}. 


Given that, it comes as no surprise that computer vision techniques can automatically analyse pictures of urban scenes and accurately determine the extent to which these scenes are, \emph{on average}, considered beautiful.  Deep learning has greatly contributed to increase these techniques' accuracy~\cite{dubey2016deep}.

However, urban planners and architects are interested in urban interventions and, as such, they wish to go beyond technologies that are only able to predict beauty scores. These interests stem from the fact that the spaces we live in can be linked with several aspects of human life such as mental health\cite{seresinhe2015quantifying}, inequality \cite{salesses2013collaborative} or cultural shifts \cite{10.3389/fphy.2018.00027}. They often called for technologies that would make easier to recreate beauty in urban design~\cite{de2008architecture}. Deep learning, by itself, is not fit for purpose. It is not meant to recreate beautiful scenes, not least because it cannot provide any explanation on why a scene is \sj{deemed beautiful, or which urban elements contributed to its beauty.}


To partly fix that, we propose a deep learning framework (which we name  FaceLift) that is able to both \emph{generate} a beautiful scene (or, better, \emph{beautify} an existing scene) and \emph{explain} why that scene is beautiful. This opens up a possibility of using technology in urban planning efforts like decision making based of subjective opinions, participatory urban planning and promotion of restorative urban design such as green spaces and walkable areas.  Through this work, we make two main contributions:

\begin{itemize}
\item We propose a deep learning framework that is able to learn whether \sj{a particular set of Google Street views are beautiful or not}, and based on that training, is able to both \emph{beautify} existing views \sj{which were deemed not to be as beautiful }and \emph{explain} which urban elements  make these views beautiful (Section~\ref{sec:framework}). \sj{The explanations come in the form of predictors of urban beauty, measured using computer vision tools.}

\item We quantitatively evaluate whether the framework is able to actually produce beautified scenes (Section~\ref{sec:evaluation}). We do so by proposing a family of \sj{five} urban design metrics that we have formulated based on a thorough review of the literature in urban planning. For all these \sj{five} metrics, the framework passes with flying colors: with minimal interventions, beautified scenes are twice as walkable as the original scenes, for example. Also, after building an interactive tool with ``FaceLifted'' scenes in Boston and presenting it to twenty experts in architecture,  we found that the majority of them agreed on three main areas of our work's impact: decision making, participatory urbanism, and promotion of restorative spaces among the general public. 

\end{itemize}
For sake of brevity, we will use the term 'Urban Scene` through out the paper to address an arbitrary Google Street View image. The image is fetched from a particular latitude and longitude point on the map. 
In the rest of the paper we explore related literature across various tracks of urban perceptions and urban beauty in Section \ref{sec:related}. We then describe in detail the Facelift framework in Section \ref{sec:framework}. The evaluation of the framework is described in detail in Section \ref{sec:evaluation} We conclude by pointing out some limitations and biases that might well guide future work (Section~\ref{sec:discussion}).




