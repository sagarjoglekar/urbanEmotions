\section{Introduction}


Whether a street is considered beautiful is subjective, yet research has shown that there are specific urban elements that are universally considered beautiful: from greenery, to small streets, to memorable spaces~\cite{alexander1977pattern, quercia2014aesthetic,salesses2013collaborative}. These elements are those that contribute to the creation of what the urban sociologist Jane Jacobs called `urban vitality'~\cite{jacobs1961death}. 


Given that, it comes as no surprise that computer vision techniques can automatically analyze pictures of urban scenes and accurately determine the extent to which these scenes are, \emph{on average}, considered beautiful.  Deep learning has greatly contributed to increase these techniques' accuracy~\cite{dubey2016deep}.

However, urban planners and architects are interested in urban interventions and, as such, they wish to go beyond technologies that are only able to predict beauty scores. They often called for technologies that would make easier to recreate beauty in urban design~\cite{de2008architecture}. Deep learning is not fit for purpose. It is not meant to recreate beautiful scenes, not least because it cannot provide provide any explanation on why a scene is beautiful. 

To partly fix that, we propose a deep learning framework (which we name  FaceLift) that is able to both \emph{generate} a beautiful scene (or, better, \emph{beautify} an existing scene) and \emph{explain} why that scene is beautiful.  In so doing, we make two main contributions:

\begin{itemize}
\item We propose a deep learning framework that is able to learn whether Google Street views are beautiful or not and that, based on that training, is able to both \emph{beautify} existing views and \emph{explain} which urban elements  make these views beautiful (Section~\ref{sec:framework}). 

\item We quantitatively evaluate whether the framework is able to actually produce beautified scenes (Section~\ref{sec:evaluation}). We do so by proposing a family of four urban design metrics that we have formulated based on a through review of the literature in urban planning. For all these four metrics, the framework passes with flying colors: with minimal interventions, beautified scenes are twice as walkable as the original scenes, for example. Also, after building an interactive tool with ``FaceLifted'' scenes in Boston and presenting it to twenty experts in architecture,  we found that the majority of them agreed on three main areas of our work's impact: decision making, participatory urbanism, and promotion of restorative spaces among the general public. 

% We find that a vast majority of participants see a general potential for the technology: 85\% of expert participants stated that it was probably better than existing tools used for \textit{``Participatory approaches on urban planning''}. 70\% expressed the  same opinions about its utilization for \textit{``decision making''}, also 70\% saw potential in its ability to \textit{``promote green cities''}.  
\end{itemize}

We conclude by pointing out some limitations that might well guide future work (Section~\ref{sec:discussion}).



