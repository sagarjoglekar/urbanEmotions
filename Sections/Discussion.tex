\section{Discussion}

\subsection{Beyond Beauty}
The pipeline is generalizable for geotagged and annotated images. The aim of this paper is to propose a pipeline with uses state of art methods in generative models to understand affects in urban images. But the pipeline can be easily extended towards a different outcome variable such as safety, public health, political status etc. 

\subsection{Limitations and biases}
Like any supervised deep learning based framework, this work is only able to learn what is present in the data. Hence the method of acquiring annotations  for urban images can introduce huge biases in the model. The current model is trained on images acquired from the study on streetscore \cite{naik2014streetscore}. However their annotation is open to general public and there is not way we can remove biases that come with culture and location, in a highly subjective effect like beauty. Moreover because the pair wise choice is simply done by clicking one of the two images, the data might have noise introduced by non-serious participants. Such biases are bound to be picked up by the deep learning model. One can argue that the preference of our model for greenery , is a form of bias in the data. 
Another Limitation of our work is in the metric formation. The computational metrics developed to capture the real urban design metrics are designed using heuristics. There needs to be more crowd and expert validation to establish the validity of their formulation. 

\subsection{Future work}
