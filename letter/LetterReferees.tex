\documentclass{paper}

\usepackage{color}
\usepackage{url}
\usepackage{framed}
\usepackage{enumerate}
\usepackage{xcolor,colortbl}
\usepackage{wrapfig}
\usepackage{tabularx}
\usepackage{graphicx}
\usepackage{mathtools}

\usepackage{balance}
\usepackage{multirow}

\usepackage{microtype}
\usepackage{xcolor}

\usepackage{amsmath}
\usepackage{makecell}
\usepackage{colortbl}
\usepackage{xstring}

\usepackage{booktabs}

\PassOptionsToPackage{hyphens}{url}\usepackage{hyperref}

\definecolor{Gray}{gray}{0.85}
\newcommand{\mytablesize}{\scriptsize}
\newcommand{\note}[1]{\textbf{\sf [\underline{SJ}: {\color{red}#1}]}}

\newenvironment{myquote}
{\definecolor{shadecolor}{rgb}{0.9,0.95,1} \begin{shaded*} \sf \em}
{\em\end{shaded*}}

\newenvironment{myquoteOrange}
{\definecolor{shadecolor}{rgb}{1,0.9,0.83} \begin{shaded*} \sf \em}
{\em\end{shaded*}}

\oddsidemargin = -0pt
\topmargin = 0pt
%\textheight = 690pt
\textwidth = 460pt

\newcommand{\ns}[1]{\textcolor{red}{NS: #1}}

\begin{document}

%\begin{quote}
%{\bf \Large ``} quote here.~{\bf \Large ''}
%\end{quote}

%\begin{myquoteOrange}

\title{Response to the reviews of ACM-TOMM submission ``FaceLift: A transparent deep learning framework recreating the urban spaces people intuitively love''}
\author{\small Sagar Joglekar, Daniele Quercia, Miriam Redi, Luca Aiello, Tobias Kauer, Nishanth Sastry}
\maketitle

We would like to express our sincere thanks to the Editors for supporting this process and the reviewers for their very detailed and constructive comments. We have worked to address all their concerns in the revised version of the manuscript. Below, we first provide a summary of our major changes, followed by a detailed response to each comment of every reviewer.

\section*{Summary of reviewer requests}

\begin{myquote}

\noindent Reviewers made certain important points which we try to address in this revision:

\begin{enumerate}
\item The description of the ``Facelift'' pipeline was not clear enough and missed on the details of the deep learning model
\item The literature review was not at par. 
\item Some clear adjustment in the description or over arching vision of the work to be either a Urban scene generator, or a framework to make the ``black-box'' more accessible
\item Clarity around the scale of the User survey
\item Over all proof-reading to sort out some spelling mistakes
\end{enumerate}

\end{myquote}

\noindent We have thoroughly revised the paper in order to follow the guidance provided. A summary response to the points above:

\begin{enumerate}


\item We improved the description of the machine learning pipeline and of the individual blocks that compose it. We added a supportive diagram of the pipeline to clarify how the two deep learning models are combined. 
%We provided samples from the scene generator to demonstrate how the generator learns 'urban scenes' 
%\textbf{\textcolor{blue}{[$\leftarrow$ Luca: I do not understand this sentence]}\textcolor{red}{[Sagar: Removed from the summary as it was not adding anything]}}. 
We also clarified the requirements and constraints of the beautification process.

\item We increased the literature survey by more than 30\% by adding references to frame our work in a broader context. 

\item We clarified what we mean by ``explaining black-box inferences''. Specifically, we rephrased the manuscript to best reflect the actual contribution of the work: instead of claiming that we are trying to completely unravel the deep learning black-box framework, we now make clearer that our objective is to make deep learning methods more relatable and consumable to the target audience.

\item We grounded our choice of the User survey scale into the previous literature that excluded a neutral option to encourage experts to avoid a de-facto position and take a stance based on their knowledge. \textcolor{blue}{\textbf{[++could not read the comments to this one++]}}
%\item We tried to address the concern about non-neutral position in the Likert scale. We make a case based on literature~\cite{moors2008exploring,Agree2012} that to avoid a de-facto position from the experts about FaceLift, it was a strategic choice to not allow a neutral position. This has shown to encourage the responders to take a position based on their expert knowledge, which is what we value greatly.

\item We revised the whole text of the paper to fix spelling and grammar mistakes.

\end{enumerate}

A detailed breakdown of the actions taken can be found next.

\section*{Requests from Reviewer 1}

\noindent Reviewer 1 had no additional comments to be addressed.

\section*{Requests from Reviewer 2}

\begin{myquote}
\noindent Comment: Related work section is not detailed enough, and I doubt the number of references is enough for a comprehensive literature review.
\end{myquote}

\noindent To address this, we did an extra round of literature survey to  contextualize this paper. We added an additional subsection to ground our work with urban design metrics in the literature. We have ended up adding more than 30\% to the literature review. 


\begin{myquote}
\noindent Comment: The description of the framework is not clear enough, and some details are missing, such as the network structure of the deep learning model.
\end{myquote}

\noindent Following the reviewer's suggestion, we expanded the description of both deep learning models: the one for classification of image beauty, and the other for reproducing urban scenes to a high degree of fidelity. We also clarified the functioning of the generative model by adding examples of how generative frameworks work. We then added a schema in Figure 5 to describe how the combination of the generative model and the classifier is used to maximize beauty in urban images. We added additional text describing the architecture of the classifier in Section \textbf{``Training a beauty classifier''}. We further cite the Generator architecture we have used, and clarify how we train this generator in Section \textbf{``Generating a synthetic beautified scene''}.
%We draw the reader's attention to the details through the relevant citations seen in the section \textbf{Training a beauty classifier} and section \textbf{Generating a synthetic beautified scene} 
%\textbf{\textcolor{blue}{[Luca: I do not understand this last sentence]}}. \textbf{\textcolor{blue}{[Luca: Clearly say in which (sub)section the new text was added]}\textcolor{red}{[Sagar: Rephrased]}}
%We also added Figure 5, which helps portray a better picture of the activation maximization method, which puts the classifier and the generator in tandem to create a beauty maximized template.


\begin{myquote}
\noindent Comment: Given an input image which is beautiful, does the framework return a more beautiful image or an ugly image? If it is the former, how to measure whether it really becomes more beautiful. 
\end{myquote}

\noindent The beautification process happens through activation maximization of the beauty classifier's output neuron. If the input image is inherently beautiful, the output neuron corresponding to the concept of ``beauty'' should be in a maximal activation state. In such a setup, the activation maximization would not be useful---the image is already beautiful and there is little room for improvement. For this reason, we test the pipeline only to turn the ugly pictures (low Trueskill scores) into beautiful ones and viceversa (as mentioned in the caption of Figure 2).
%This prevents these situations and ensures that the input images are either inherently un-aesthetic or inherently beautiful.
In the revised version, we have added the following text in Section \textbf{``Generating a synthetic beautified scene''}:

\begin{myquoteOrange}
Maximizing beauty of an already beautiful image would yield a saturated template $\hat{I_j}$. For this reason, to generate an image that maximizes the beauty neuron in the classifier $C$, the network needs to be supplied with an image that lies in the class $y_i$. The constraint is reverse for maximizing for class $y_j$.
\end{myquoteOrange}


\section*{Requests from Reviewer 3}

\begin{myquote}
\noindent Comment: There are two main contributions at this paper. First is generating better urban scenes and second is able to explain deep learning framework with detail. But I think these two aspects are different, and address both of them will not highlight the main topic or innovation well. Actually, I cannot find, or the authors do not explicitly point out how to make ``black-box'' of CNN more apparently with detail description. So I suggest the authors only highlight one of them.
\end{myquote}

%\ns{Your answer below seems to miss the point - The reviewer seems to be saying you are trying to achieve two \emph{different} things, and you are not doing a good job highlighting both of them, so just focus on one and not the other. I tend to agree -- because you have the extensive eval with the experts, you are doing a good job of showing that the beautification works well. But it is not clear how the black box nature of neural networks is being lifted. Fig 9 simply says what aspects of an urban scene change as a result of FaceLifting, not \textbf{how}. So may be reposition within the text to remove or de-emphasise the claim that you are lifting the black box nature. And the response below would then be - ``Thank you for the suggestion. We have focussed the paper on the FaceLift Pipeline, and how it achieves its ultimate goal of  explainable beautification of urban scenes. Here's the list of changes we have done (intro + wherever else)''}
\noindent This is a valid concern. We have now de-emphasised the `black-box' claim \textcolor{blue}{\textbf{[++could not read the rest of the text here++]}}
%but made sure that we describe the pipeline's capacity to make deep learning insights consumable and interpretable for urban design. This claim is supported through our expert survey. To that extent, we have rephrased the \textbf{Abstract}. \textbf{Introduction} and the \textbf{Conclusion} sections.
%We have focused the paper on the FaceLift Pipeline, and how it achieves its ultimate goal of  explainable beautification of urban scenes. Our emphasis is on FaceLift pipeline as a tool for urban planners and the urban design specific insights that the pipeline is able to produce for decision making. We modify the abstract, introduction and discussion to position the paper in this context. We stress that the transparency of this pipeline is attained through the explanations around the transformations. 
%\textbf{\textcolor{blue}{[Luca: I feel this response is too vague. The response to the Editor about this comment (beginning of the letter) is a little more to the point.]}\textcolor{red}{[Sagar: Rephrased]}}


\begin{myquote}
\noindent Comment: About ``Q4 Do architects and urban planners find it useful?'', as the results shown in Table 6, I find that the rating from ```definitely not'' to ``definitely'' without a ``neutral'' option, and this will lead to bias.
\end{myquote}

\noindent The question of providing a neutral option has been debated for decades, with convincing arguments on both sides. Previous work~\cite{Agree2012,moors2008exploring} showed that neutral choices end up commonly used to express \textit{lack of knowledge or indifference}. All our participants were experts in their respective fields of urban design, data visualization and architecture. Hence we consciously wanted them to express an opinion about the utility of such a tool in their practice. We clarified this aspect with the following text:

\begin{myquoteOrange}
\textit{Since our respondents were experts in different areas, we wanted them to express an opinion about the utility of such a technology in their practices, and only give non-neutral responses. In accordance with this constraint, we designed the survey based on a non-neutral response Likert scale, as critical studies suggest~\cite{Agree2012,moors2008exploring}}
\end{myquoteOrange}


\begin{myquote}
\noindent Comment: In Fig.8 there are five urban design metrics, but in the section of abstract and Table 4, there are only four metrics: walkability, green, openness, and visual complexity. 
\end{myquote}

\noindent We thank the reviewer for pointing this out. We have added back the fifth metric of `landmarks' into the paper. The metric is described in Table 4 and referred to when needed. Past studies have shown the role of landmarks in shaping the perception of `goodness' and `memorability' \textcolor{blue}{\textbf{[++could not make the new term here++]}} of a city, at the macroscopic level of neighborhoods~\cite{quercia2014aesthetic,lynch1960image}.
%Unfortunately, our setup was not suitable to do any hypothesis testing along these lines at the microscopic level of individual urban scenes. That is because our model has a limited capability to detect diverse landmarks with high accuracy. We would like investigate further in the follow-up extension of this work.

\section*{Some of the positive comments...}

\begin{myquote}

\noindent Reviewers noted many positive aspects about this paper. Common across all three reviews were an appreciation of the usefulness of the work for the community as well as the innovation in this piece of work. We are grateful to Reviewer 1 for nominating this work for best paper and commending the work.  

\end{myquote}

We want to express our sincere thanks to the Editors and to the Reviewers for all the constructive feedback as well as the positive comments above, and hope they will find the new version of the paper much improved.

\bibliographystyle{abbrv}
\bibliography{letter}

\end{document}
