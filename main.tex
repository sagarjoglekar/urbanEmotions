%\documentclass[a4paper]{article}
\documentclass[format=acmsmall, review=false, screen=true]{acmart}
\usepackage{booktabs} % For formal tables

\usepackage[ruled]{algorithm2e} % For algorithms
\renewcommand{\algorithmcfname}{ALGORITHM}
\SetAlFnt{\small}
\SetAlCapFnt{\small}
\SetAlCapNameFnt{\small}
\SetAlCapHSkip{0pt}
\IncMargin{-\parindent}

% !TEX root = ../kdd2014.tex

\usepackage{xcolor}
%\usepackage{amsmath}
%\usepackage{verbatim}
%\usepackage{graphics}
\usepackage{subfig}
\usepackage{graphicx}
\usepackage{graphics}
\usepackage{multicol}
\usepackage{url}
%%decrease spacing 
\usepackage{times}
\usepackage{comment}
\usepackage{balance}
%
%\newtheorem{thm}{Theorem}
%\newtheorem{corr}{Corollary}
%\newtheorem{lemma}{Lemma}
%\newtheorem{pty}{Property}
 
%\definecolor{comment}{rgb}{0.0, 0.5, 1.0}
\newcommand{\red}[1]{\textcolor{red}{#1}}
\newcommand{\blue}[1]{\textcolor{blue}{#1}}
\newcommand{\green}[1]{\textcolor{green}{#1}}
\newcommand{\ns}[1]{\green{#1}}
\newcommand{\mr}[1]{\blue{#1}}
\newcommand{\sj}[1]{\green{#1}}
\newcommand\dq[1]{\red{#1}}
%\newcommand\tabhead[1]{\small\textbf{#1}}
\newcommand\fref[1]{Figure~\ref{#1}}
\newcommand\tref[1]{Table~\ref{#1}}
\newcommand\Sref[1]{\S\ref{#1}}
\newcommand\etal{~\textsl{et al.}} % use as Sastry\etal~\cite{xxx}
\newcommand\tabhead[1]{\small\textbf{#1}}
\newcommand{\rotate}[1]{\rotatebox[origin=c]{90}{#1}}

\newcommand*{\thead}[1]{\multicolumn{1}{c}{\bfseries #1}}


%Numbers
\newcommand{\one}{({\it i}\/)}
\newcommand{\two}{({\it ii}\/)}
\newcommand{\three}{({\it iii}\/)}
\newcommand{\four}{({\it iv}\/)}
\newcommand{\five}{({\it v}\/)}
\newcommand{\six}{({\it vi}\/)}
\newcommand{\seven}{({\em vi}\/)}

\let\olditemize\itemize
\renewcommand{\itemize}{
  \olditemize
  \setlength{\itemsep}{1pt}
  \setlength{\parskip}{0pt}
  \setlength{\parsep}{0pt}
}
\let\olddescription\description
\renewcommand{\description}{
  \olddescription
  \setlength{\itemsep}{1pt}
  \setlength{\parskip}{0pt}
  \setlength{\parsep}{0pt}
}
\let\oldenumerate\enumerate
\renewcommand{\enumerate}{
  \oldenumerate
  \setlength{\itemsep}{1pt}
  \setlength{\parskip}{0pt}
  \setlength{\parsep}{0pt}
}

%
% --- Author Metadata here ---
%\conferenceinfo{WOODSTOCK}{'97 El Paso, Texas USA}
%\CopyrightYear{2007} % Allows default copyright year (20XX) to be over-ridden - IF NEED BE.
%\crdata{0-12345-67-8/90/01}  % Allows default copyright data (0-89791-88-6/97/05) to be over-ridden - IF NEED BE.
%% --- End of Author Metadata ---


%\permission{Copyright is held by the International World Wide Web Conference Committee (IW3C2). IW3C2 reserves the right to provide a hyperlink to the author's site if the Material is used in electronic media.}
%\conferenceinfo{WWW 2016,}{Apr 3--7, 2016, Perth, Australia.} 
%\copyrightetc{ACM \the\acmcopyr}
%\crdata{978-1-4503-3469-3/15/05. \\
%http://dx.doi.org/10.1145/2736277.2741671}

\clubpenalty=10000 
\widowpenalty = 10000


% \author{Sagar Joglekar}
% \orcid{1234-5678-9012-3456}
% \affiliation{%
% 	\institution{King's College}
% 	\department{Department of Informatics}
% 	\city{London}
% 	\country{United Kingdom}}

% \author{Miriam Redi}
% \affiliation{%
% 	\institution{Nokia Bell Labs}
% 	\city{Cambridge}
% 	\country{United Kingdom}}

% \author{Nishanth Sastry}
% \affiliation{%
% 	\institution{King's College}
% 	\department{Department of Informatics}
% 	\city{London}
% 	\country{United Kingdom}}


%\numberofauthors{3}
%
%
%\author{
%\alignauthor
%Sagar Joglekar\\\affaddr{King's College London, UK}\\
%\affaddr{sagar.joglekar@kcl.ac.uk}
%\alignauthor
% Nishanth Sastry\\\affaddr{King's College London, UK}\\
%\affaddr{nishanth.sastry@kcl.ac.uk}
%\alignauthor
%Miriam Redi\\\affaddr{Bell Labs, UK}\\
%\affaddr{miriam.redi@nokia.com}
%}
%
%
%\def\sharedaffiliation{%
%\end{tabular}
%\begin{tabular}{c}}
%
%\author{
% \alignauthor Sagar Joglekar \\
%% \email{firstname.lastname}@kcl.ac.uk \\
% \affaddr{King's College London, UK}\\ 
% \alignauthor Nishanth Sastry \\
%% \email{firstname.lastname}@kcl.ac.uk \\
%% \sharedaffiliation
% \affaddr{King's College London, UK}\\ 
% 
% \alignauthor Miriam Redi \\
%% \email{firstname.lastname}@nokia.com\\
% \affaddr{Bell Labs, UK}\\
%}



%\usepackage{acm}

%For some reason this format only inserts abstract if it is before begin document :/


\begin{document}
\title[FaceLift]{FaceLift: A transparent deep learning framework recreating the urban spaces people intuitively love}


\author{Sagar Joglekar}
\affiliation{%
	\institution{King's College London, Department of Informatics}}
\email{sagar.joglekar@kcl.ac.uk}

\author{Daniele Quercia}
\affiliation{%
	\institution{Nokia Bell Labs, Cambridge, UK}}
\email{daniele.quercia@gmail.com}

\author{Miriam Redi}
\affiliation{%
	\institution{Nokia Bell Labs, Cambridge, UK}}
\email{miriam.redi@gmail.com}

\author{Luca Maria Aiello}
\affiliation{%
	\institution{Nokia Bell Labs, Cambridge, UK}}
\email{luca.aiello@nokia-bell-labs.com}

\author{Tobias Kauer}
\affiliation{%
	\institution{Nokia Bell Labs, Cambridge, UK}}
\email{tobias.kauer@fh-potsdam.de}

\author{Nishanth Sastry}
\affiliation{%
	\institution{King's College London, Department of Informatics}}
\email{nishanth.sastry@kcl.ac.uk}
\makeatletter

\def\@copyrightspace{\relax}
%\makeatother
\keywords{Deep learning, Urban Design,
	Generative models, Urban beauty, Explainable models}

\begin{abstract}
	There has been an explosive growth of deep learning technologies and their competency in the recent years, resulting in cross disciplinary use cases of deep learning enabled tools. In the area of computer vision and urban informatics, deep learning techniques have recently been used to predict whether urban scenes are likely to be considered beautiful, and it turns out that these techniques do so quite accurately. However, the technology falls short when it comes to generating actionable insights for AI assisted urban design. To support urban interventions, one needs to go beyond \emph{predicting} beauty, and tackle the challenge of \emph{recreating} beauty and \emph{explaining} the predictors of beauty. And for these explanations to be of any use to the target audience of such tools, they need to be grounded in the literature and language of the target users. Unfortunately, deep learning techniques have not been designed with that challenge in mind. Given their ``black-box nature'', these models cannot be directly used to explain why a particular urban scene is deemed to be beautiful. To partly fix that, we propose a deep learning framework (which we name  FaceLift) that is able to both \emph{beautify} existing Google Street views and \emph{explain} which urban elements make those transformed scenes beautiful, in the vocabulary of urban design science. To quantitatively evaluate our framework, we cannot resort to any existing metric (as the research problem at hand has never been faced before) and need to  formulate new ones. These new metrics should ideally capture the presence (or absence) of elements that make urban spaces great. They ideally should also be computable using current computer vision techniques. Upon a review of the urban planning literature, we identify \textsl{five} main metrics: walkability, green spaces, openness, landmarks and visual complexity.  For all the five metrics, the beautified scenes meet the expectations set by the literature on what great spaces tend to be made of. The transformations and their explanations are also found to be very helpful in understanding interventions for beautification, which we validate using a 20-participant expert survey. These results suggest that, in the future, as our framework's components are further researched and become better and more sophisticated, it is not hard to imagine technologies that will be able to accurately and efficiently support architects and planners in the design of the spaces we intuitively love.
\end{abstract}

\keywords{Deep learning, Generative networks, Urban Beauty, Computer Vision}

\maketitle

\section{Introduction}


Whether a street is considered beautiful is subjective, yet research has shown that there are specific urban elements that are universally considered beautiful: from greenery, to small streets, to memorable spaces~\cite{alexander1977pattern, quercia2014aesthetic,salesses2013collaborative}. These elements are those that contribute to the creation of what the urban sociologist Jane Jacobs called `urban vitality'~\cite{jacobs1961death}. 


Given that, it comes as no surprise that computer vision techniques can automatically analyse pictures of urban scenes and accurately determine the extent to which these scenes are considered, \emph{on average}, beautiful.  Deep learning has greatly contributed to increase these techniques' accuracy~\cite{dubey2016deep}.

However, urban planners and architects are interested in urban interventions and, as such, they wish to go beyond technologies that are only able to predict beauty scores. They have often called for technologies that would make easier to recreate beauty in urban design~\cite{de2008architecture}. Deep learning, by itself, is not fit for purpose. It is not meant to recreate beautiful scenes, not least because it cannot provide any explanation on why a scene is deemed beautiful, or which urban elements are predictors of beauty.


To partly fix that, we propose a deep learning framework (which we name  FaceLift) that is able to both \emph{generate} a beautiful scene (or, better, \emph{beautify} an existing scene) and \emph{explain} why that scene is beautiful. In so doing, we make two main contributions:

\begin{itemize}
\item We propose a deep learning framework that is able to learn whether a particular set of Google Street Views (urban scenes) are beautiful or not, and based on that training, the framework is then able to both \emph{beautify} existing views and \emph{explain} which urban elements  make them beautiful (Section~\ref{sec:framework}). These explanations are automatically extracted with computer vision tools. 

\item We quantitatively evaluate whether the framework is able to actually produce beautified scenes (Section~\ref{sec:evaluation}). We do so by proposing a family of five urban design metrics that we have formulated based on a thorough review of the literature in urban planning. For all these five metrics, the framework passes with flying colors: with minimal interventions, beautified scenes are twice as walkable as the original scenes, for example. Also, after building an interactive tool with ``FaceLifted'' scenes in Boston and presenting it to twenty experts in architecture,  we found that the majority of them agreed on three main areas of our work's impact: decision making, participatory urbanism, and the promotion of restorative spaces.
\end{itemize}






\section{Related Work}
\label{sec:related}
Previous work has focused on collecting ground truth data about how people perceive urban spaces, on predicting urban qualities from visual data, and on generating synthetic images that enhance a given quality (e.g., beauty). 


\mbox{}\\
\noindent
\textbf{Ground truth of urban perceptions.} So far the most detailed studies of perceptions of urban environments and their visual appearance have relied on personal interviews and observation of city streets: for example, some researchers relied on annotations of video recordings by experts~\cite{sampson04seeing}, while others have used participant ratings of simulated (rather than existing) street scenes~\cite{lindal2012}. The web has recently been used to survey a large number of individuals. Place Pulse is a website that asks a series of binary perception questions (such as `Which place looks safer [between the two]?') across a large number of geo-tagged images~\cite{salesses2013collaborative}. In a similar way, Quercia \emph{et al.} collected pairwise judgments about the extent to which urban scenes are considered quiet, beautiful and happy~\cite{quercia2014aesthetic}. They were then able to analyze the scenes together with their ratings using image-processing tools, and found that the amount of greenery in any given scene was associated with all three attributes and that cars and fortress-like buildings were associated with sadness. Taken all together, their results pointed in the same direction: urban elements that hinder social interactions were undesirable, while elements that increase interactions were the ones that should be integrated by urban planners to retrofit cities for greater happiness. 

\mbox{}\\
\noindent
\textbf{Deep learning and the city.} Computer vision techniques have increasingly become more sophisticated. Deep learning techniques, in particular, have been recently used to accurately predict urban beauty~\cite{dubey2016deep,seresinhe2017using}, urban change~\cite{naik2017computer}, and even crime~\cite{DeNadai16}.

\mbox{}\\
\noindent
\textbf{Generative models.} Deep learning has recently been used not only to analyze existing images but also to generate new ones. Ngyuen \emph{et al.}~\cite{nguyen2016synthesizing} used generative networks to create a natural-looking image that maximizes a specific neuron. In theory, the resulting image is the one that ``best activates'' the neuron under consideration (e.g., that associated with urban beauty). In practice, it is still a synthetic template that needs further processing to look realistic. \mbox{} \\

\mbox{}
To sum up, a lot of work has gone into collecting ground truth data about how people tend to perceive urban spaces, and into building accurate predictions models of urban qualities. However,  little work has gone into models that generate realistic urban scenes and that offer human-interpretable explanations of what they generate. 



\section{FaceLift Framework}
\label{sec:framework}

\begin{figure}[t!]
	\centering
	\includegraphics[width=0.7\columnwidth]{Plot/Trueskill.png}
	\caption{Distribution of ordinal scores for images with at-least 4 votes. The red and green line represent the threshold below and above which images are tagged Ugly or beautiful. Images in between are dropped for separability }
	\label{fig:Trueskill}
\end{figure}

The goal of FaceLift is to take as input a geo-located urban scene and give as output its transformed (beautified) version. 

\begin{figure*}[t!]
	\centering
	\hspace*{-5mm}
	\subfloat[]{
		\includegraphics[width=0.3\textwidth, height = 5cm ]{Plot/rotationalSim.png}
		\label{fig:rotSim}
	}
	\subfloat[]{
		\includegraphics[width=0.3\linewidth, height = 5cm ]{Plot/transSim.png}
		\label{fig:transSim}
	}
	\vspace{-0.4cm}
	\caption{ Fig \ref{fig:rotSim} shows an example set of images showing similarity of streetview scapes, when the camera is rotated by a small angle. Fig \ref{fig:transSim} shows the translational similarity where the angle is less than the established bound $\rho$ }
	\vspace{-0.4cm}
\end{figure*}

To begin with, we need highly curated training data with labels reflecting urban beauty. We start with the  Place Pulse dataset that contains 100k Google Street Views across 56 cities around the world~\cite{dubey2016deep}. These are labeled in terms of whether the corresponding places are likely to be perceived beautiful, depressing, rich, and safe. We focus only on those scenes that are labeled in terms of beauty and that have at least three judgments. This leave us with roughly  20,000 scenes. To transform judgments into beauty scores, we use the TrueSkill algorithm~\cite{herbrich2007trueskill}, which gives us a way of partitioning the scenes into two sets (Figure \ref{fig:Trueskill}): one containing beautiful scenes, and the other containing ugly scenes. The resulting set of scenes is too small for training any deep learning module without avoiding over-fitting though. As such, we need to augment such a set. We do so in two ways. First, we feed each scene's location into the Google Streetview API to obtain  the snapshots of the same location at different camera angles (i.e., at $\theta \in {-30^{\circ}, -15^{\circ} , 15^{\circ} , 30^{\circ} }$). However, the resulting dataset could not still be enough. Therefore, again, we feed each scene's location into the Google Streetview API, but we now do so to obtain other scenes at  distance $d \in \{10,20,40,60\}$ meters.  This will greatly expand our set of scenes at the price of introducing scenes whose beauty scores have little to do with the original scene's score. To fix that, we take only the scenes that are similar to the original one. To compute that similarity between two scenes,
we represent the scenes with visual features derived from the FC7 and compute the similarity between the two feature vectors~\cite{}. For all  scenes at distance $d \in \{10,20,40,60\}$ meters,  we take only those whose similarity with the original scene is above a threshold. In a conservative fashion, we go back to the first augmentation step done by simply rotating the camera and choose that threshold to be the median similarity between rotated and original scenes. To make sure this additional augmentation step has not introduced unwanted noise, we consider  two sets of scenes: one containing those that have been taken during the step (\emph{taken-set}), and the other containing those that have been filtered away (\emph{filtered-set}). 

\begin{figure}[ht]
	\centering
	\includegraphics[width=\columnwidth]{Plot/SimilarityPlacesPrevalence.png}
	\caption{Prevalence plot of types of scenes prevalent in Similar images compared to dissimilar ones.}
	\label{fig:augmentationSimilarity}
\end{figure}

Each scene is labeled with PlacesNet~\cite{zhou2014learning} and is represented with the five most confident scene labels. The labels are aggregated  at a set level by computing each label's frequency on the \emph{taken-set} minus that on the \emph{filtered-set}: 
$ \textrm{prone}(label)= fr(label,\textrm{\emph{taken-set}}) - fr(label,\textrm{\emph{filtered-set}}).$
The resulting quantity reflects the extent to which a scene with a given label is prone to be augmented or not. As one would expect, from Figure~\ref{fig:augmentationSimilarity}, we find that scenes that contain highways, fields and bridges can be augmented at increasing distances while still resembling the original scene; by contrast, scenes that contain gardens, residential neighborhoods , plazas, and skyscrapers cannot be easily augmented. 


Having this highly curated set of urban scenes, we are now ready to describe the steps ....




 

%To develop a better way of augmenting images which can be transferred with scores of the original annotated ones, we make one heuristic assumption : "\textbf{[A1]}\textit{the composition of a StreetView image does not change considerably for small rotations of the camera angle}". This assumption was tried and tested over several samples both manually and using image similarity measures. An example of one such sample can be seen in Figure \ref{fig:rotSim}. This assumption allows us to do a basic expansion of our dataset without adding a lot of noise. However we cannot to a similar assumption when it comes to translation of camera. 



%All images in the PlacePulse dataset are taken with a default camera rotation which depends on the location. The Streetview API allows to specify the preferred camera rotation angle. This gives us the opportunity to take snapshots of the same location at different camera angles. We rotate the camera across different values $\theta \in {-30^{\circ}, -15^{\circ} , 15^{\circ} , 30^{\circ} }$ and derive a first set  $R=\{R(I)_{\theta} \forall I \in X\}$ , which consists of images acquired by rotating the camera angle for each annotated image in $X$. Following \textbf{A1}, For these images, the beauty score of each rotated image $R(I)_{\theta}$ can be safely transferred from $I$. 
%First set  is $R(I_i) \forall I_i \in X$ , which consists of images acquired b	y just rotating the camera angle for a given annotated image. We rotate the camera across different values $\theta \in {-30^{\circ}, -15^{\circ} , 15^{\circ} , 30^{\circ} }$. For these images, the beauty score can be safely transferred from $i$. 

%Next, for each image in $X$, we translate the location of the Streetview camera: we select points on the map at a distance of $d \in \{10,20,40,60\}$ meters and acquire the resulting set of images $T=\{T(I)_d \forall I \in X\}$%, by translating the location of Streetview camera a near geographic vicinity of a StreetView image $I_i$ in $X$. We by translating the streetview  camera at distances of 10, 20 40 an,d 60 meters. Although possibly very similar, transferring the beauty score from $I$ to each $T(I)_d$ might result in very noisy data. To understand the extent to which beauty scores can be transferred from images to their translated version, we use a smart augmentation technique. 


%%In a nutshell, this technique computes the similarity between the translated and the original image, and transfers the beauty scores only if the similarity is acceptably high.
%To do so, we represent each images from both sets using visual features extracted from the FC7 layer of PlacesNet \cite{zhou2014learning}. We then calculate the cosine similarities $ S_t = \{s(I,T(I)_d) \forall I \in X \}$
%between each original image $I$ and all images in the augmented set $T(I)_d$. 
%We also calculate another set of cosine similarities  $ S_r = \{s(I,R(I)_{\theta}) \forall I \in X \}$ between rotated and origianl images. We define a similarity bound as the median similarity betwen rotated and original images.
%\begin{equation}
%\rho = median(S_r) \text{ where }{S_r} = \{s(I,R(I)_{\theta}) \forall I \in X \}
%\label{eq:bound}
%\end{equation}
%Following the assumption \textbf{[A1]} , we only transfer beauty scores to translated images who look as similar to the original as their rotated counterparts: $s(I,T(I)_d) < \rho$. We discard translated images not fullfilling this requirement and retain the resulting images in the smartly translated image set $\hat{T}$. %namely only to those translated images whose similarity score to the original image  rotation of camera. 
%
%






%%We present here Facelift, an end-to-end  framework for
% image beautification. The framework embeds a model trained on a set of urban images annotated with beauty scores. It takes as input a geolocated urban image and gives as output its transformed (beautified) version. %The framework The current process concentrates for Beautification of urban images for the sake of this study, but the framework 
% Although we refer here to specific urban properties (i.e. beauty) and datasets, Facelift is generalizable to any labeled dataset of geolocated images. %The components work equally effective given that the principles of training a deep-learning model are followed. 
%%We present here a framework for transforming natural geolocated images from one category to another. The current process concentrates for Beautification of urban images for the sake of this study, but the framework is generalizable for any annotated urban image dataset. The components work equally effective given that the principles of training a deep-learning model are followed. 
%
%For the sake of berevity, we summarise the notations used in Table \ref{notations} and the framework steps in Figure \ref{fig:framework}. 
%% \textbf{[add notations of T and R]}


\begin{table}[t]
	\resizebox{\linewidth}{!}{
		\begin{tabular}{l|p{8cm}}
			\textbf{symbol} & \textbf{stands for}\\
			$X$    & Georeferenced urban image dataset \\
			$I_i$    & Georeferenced image  $\in X $ \\
			$Y$    & Annotations classes for $X$ (e.g. beautiful/ugly)\\
			$y_i$    & Class in $Y$ (e.g. beautiful) \\
			$\hat{I_j}$ & Template image \\
			$I'$ & Target Image \\
			$C$ & Image Classifier \\
			$R$ & Images acquired by rotating street view camera \\
			$T$ & Images acquired by translating street view camera\\
			$\rho$ & Similarity bound below which smart augmentation chooses translated images \\
			& \\
			\textbf{term} & \textbf{stands for}\\
			\textit{Template Image} $\hat{I_j}$    & A synthetic transformation of input image $I$ towards the class $y_j$ \\
			\textit{Target Image} $I'$    & The natural image which is most visually similar to the template image \\
		%	\textit{ Data Clustering}    & A process which groups images in $X$ according to visual similarity (e.g urban vs rural)\\
			\textit{Data Augmentation}    & A process of data expansion which looks for images taken in the surroundings of the georeferenced images in $X$\\
			\textit{Classifier}   & A deep-learning framework that is able to classify images into one of the classes in $Y$\\
			\textit{Generator} $(GAN)$    & A deep-learning based image generator \\% framework to produce images similar to the ones in  $X$\\
			$DGN-AM$    & A framework that, given the GAN and the Classifier, transforms an input image into the template image.\\
	\end{tabular}}
	\caption{Notations and Terms.}\label{notations}
\end{table}

 \begin{figure*}[ht]
	\centering
	\includegraphics[width=2\columnwidth]{Plot/UrbanEmotionspipeline.png}
	\caption{Architecture of the Beautification framework}
	\label{fig:framework}
\end{figure*}

%In general terms, the framework allows anyone with an arbitrary set of \emph{geolocated} images $ X = { I_1, I_2 ... . I_n  }$ annotated in classes $Y = {y_1 , y_2 , ... ,y_k}$, to transform natural images between classes: the algorithm can transform an  image $I_i$ belonging to class $y_i \in Y$ , to image $I_j$ from class $y_j \in Y$. Both $I_i$ and $I_j$ are natural, non-synthetic images. Despite having another \emph{meaning} (i.e. category), $I_j$ maintains the structural characteristics of $I_i$ (e.g. point of view, layout).  This allows  to visually reason about the discriminative properties between classes $y_i , y_j \in Y$, and visually understand the salient characteristics that drive a classifier to distinguish between  classes $y_i,y_j$. %The questions pertaining to what the network learns semantically have been explored for popular use cases \cite{mao2014explain,karpathy2015deep}  but still remain largely unexplored for intangible classes representing concepts like beauty, sentiment etc.  In this paper, we apply this general scheme to the specific problem of predicting beauty and explaining the changes that influence the beautification process. ---> TO RELATED WORK
\par 
%The transformation framework consists of three phases (see Fig. \ref{fig:framework}). In the first phase, we classify images from $X$ into the corresponding categories $Y$ with high accuracy , using a convolutional neural network $C$. In our case, $y_i$  and  $y_j$ are the beautiful/ugly classes.
%In the second phase, we transform am image from class $y_i$ to class $y_j$, using Generative Adversarial Networks\cite{radford2015unsupervised}. The output of this phase is a synthetic image $\hat{I_j}$, which summarizes the basic traits of the destination class $y_j \in Y$. The last phase matches the synthetic image $\hat{I_j}$,  with the closest natural image in $X$. Finally, to quantitatively reason about the beautification process, we perform aggregated analysis of the differences between original images and resulting target images.% matched natural image is then used to reason about urban design metrics that the network learnt to associate with beauty. 
%We do this by quantifying the presence of 5 urban design metrics in uglified and beautified images.

%For the rest of this section, we would delve deeper into the specifics of the image beautification framework. 

%\subsection*{Step 1 Classifying Beauty}
%We design here a classifier $C$  able to correctly assess the beauty category $y_i$ of an image in $X$ using a deep learning network. To reliably train a convolutional neural netowrk we need first make sure we have enough reliable data to train the classifier \textbf{[REF]}. We do this by augmenting the available geolocated image data. 

%\mbox{} \\
%\noindent
%\textbf{Dataset and Beauty Judgments.}
%\label{sec:label}
%Our seed dataset comes from Place pulse, a research work on urban affective dimensions \cite{dubey2016deep}. The dataset in total contains 100k images across 56 cities around the world from Google StreetView\footnote{https://maps.googleapis.com/maps/api/streetview}. Images are annotated through pair-wise comparison for qualities such as beauty, depression, richness , safety etc. For the purpose of our work, we use the beauty judgements. To train our classifier $C$ to detect beauty %in terms of 
%categories $Y$, we need to transform pairwise votes into absolute scores, then discretize absolute scores into a finite set of categories $y_i$. We transform the pairwise votes %from the dataset are then transformed 
%into ordinal scores using the TrueSkill \cite{herbrich2007trueskill} algorithm.  To ensure reliability of absolute judgements, we filter out images with less than 3 votes.
%To discretize the resulting scores, we heuristically partition the data into two classes with maximum separation: beautiful and ugly. Figure \ref{fig:Trueskill} shows the distribution of Trueskill score estimates with the threshold scores at which we decide beauty or ugly class boundary. 



%\mbox{} \\
%\noindent
%\textbf{Data Augmentation similarity bound.}
%Despite over a hundred thousand images in the original data, only 20,000 has more than 3 judgements.
%This is non-ideal to train classifiers with substantial number of parameters such as convolutional neural network,
%since smaller data size implies that a machine learning model has a risk of over-fitting \textbf{[REF]}. We choose to augment the dataset by exploiting the geo-located nature of the image dataset. We also take advantage of the fact that urban places in close proximity look quite similar to each other \cite{parislooklikeparis}  
%%\textbf{[There was a reference here from Daniele]}.
%To develop a better way of augmenting images which can be transferred with scores of the original annotated ones, we make one heuristic assumption : "\textbf{[A1]}\textit{the composition of a StreetView image does not change considerably for small rotations of the camera angle}". This assumption was tried and tested over several samples both manually and using image similarity measures. An example of one such sample can be seen in Figure \ref{fig:rotSim}. This assumption allows us to do a basic expansion of our dataset without adding a lot of noise. However we cannot to a similar assumption when it comes to translation of camera. 
%we derive two new sets of augmented images. 
%\textbf{[Explain this assumption better - give example, refer to \ref{fig:rotSim} ]}.
%With an assumption that "\textbf{[A1]}\textit{rotation of camera, keeping the location constant, does not change the composition of the image considerably}" we now have two sets of images. 


%\begin{figure*}[ht]
%	\centering
%	\includegraphics[width=1.25\columnwidth]{Plot/AugmentationExample.png}
%	\caption{Example images showing similarity of streetview scapes, when the camera is rotated by a small angle. The translational example shows similarity of images where the angle is less than the established bound $\rho$}
%	\label{fig:augmentationExample}
%\end{figure*}







\mbox{} \\
\noindent
\textbf{The Beauty Classifier.}
Once we have enough data, we train a deep convolutional network %so as 
to classify images into the $Y$ beauty classes. One may use several successful deep convolutional neural network architectures, which work for other use cases like AlexNet \cite{krizhevsky2012imagenet} , PlacesNet \cite{zhou2014learning} or GoogLeNet \cite{szegedy2015going}.  For our paper we use CaffeNet which is a modified version of AlexNet. This trained classifier is a important component in the next phase, which is generation of images. 

We employ the above observations to train the beauty classifier model. We start with  the base dataset of 20k images ($X$), as described in Section 
\ref{sec:label}. We then progressively augment the data: first with rotation across 5 angles ($X$,$R$), then rotation with uniform translation for all images ($X$,$R$,$T$), and then rotation and translation for only images which satisfy similarity bound as evaluated as shown in Equation \ref{eq:bound} ($X$,$R$,$\hat{T}$). 
We then train a Convolutional neural net model based on AlexNet architecture \cite{szegedy2015going} on each of these augmentated datasets. The training is done on 70\% split of the data and tested on the 30\%. 

We see considerable improvement in classifier accuracy \ref{tab:classifier}, with the best model performing at 73\% accuracy for classifying images in two classes of Beauty and Ugly. 
This model represents the concept of beauty, learned from annotated and augmented streetview images. 


\begin{table}[h]
	\centering
	\begin{tabular}{|c|c|}
		\hline
		\textbf{Policy} & \textbf{Accuracy (Percentage)}\\
		\hline
		No augmentation ($X$) & 63 \\
		\hline
		Rotation only & 68 \\
		\hline
		Rotation + translation  & 64 \\
		\hline
		Rotation + Smart Translation & 73.5 \\
		\hline
		
		\hline
	\end{tabular}
	\caption{Performance differences based on different augmentation policies}
	\label{tab:classifier}
    \vspace{-10mm}
\end{table}



\subsection*{Step 2 Generating Images}
\par 
We now want to design a framework to transform any image $I_i$ of class $y_i$ into template image $\hat{I_j}$ (as shown in Figure \ref{fig:framework}), namely a synthetic version of the original image, with added features and motifs that maximize class $y_j$. 
To produce the template image $\hat{I_j}$, we need the following components in place:
\begin{itemize}
	\item {\textit{Classifier}}. We need a deep classifier $C$ able to classify $I$ into $Y$, i.e. between ugly and beautiful images. %This implies that it has learnt the abstract concept of beauty to some extent.
	
	\item \textit{Generator}. We train a generative adversarial network (GAN) which can generate an approximate natural looking image drawn from distribution of a particular class of images, similar to the one  in \cite{dosovitskiy2016inverting}. 
	
	\item \textit{Activation Maximization}. We plug in the GAN and the classifier network into an Activation Maximisation (AM) framework. Given these components, an input image $I$, and a target beauty class $y_i$, the AM transforms $I$ in an ideal image $\hat{I_j}$ ( that maximizes the activation for beauty class $y_i$).	
\end{itemize}
 
 We have described the design and performace of Classifier in Sec. \ref{sec:classifier}. We will delve deeper into the other two below
 
\mbox{} \\
\noindent
\textbf{Generator.}
 Generative Adverserial networks are an extremly useful tool when it comes to generating samples from a learned distribution\cite{radford2015unsupervised}. GANs consist of a %by learning a 
 pair of networks where the \textit{generator} %learns the distribution of sample space and 
 generates image samples similar to an input space using de-convolutional layers, and the \textit{discriminator} learns to discriminate between natural images from the training set and synthetic images generated by the generator. %The problem is a min-max arrangement where we want to maximize error in discriminator by minimizing error between generated and natural images, there by generating samples that confuse the discriminator. But 
Since GANs are known to be very tricky to train \cite{gulrajani2017improved}  we first try to use a pre-trained GANs on Imagenet from \cite{nguyen2016synthesizing}.However, because of the vast difference between images in Imagenet and Streetview images, our initial results were not very optimal. We therefore retrained the generator on the StreetScore dataset. %we used for Classification model. 
This improved the visual quality of the generated images considerably.%GAN performance considerably and it started generating images which do not entirely resemble natural streetview images, but look like paintings of these scenes. 

\mbox{} \\
\noindent
\textbf{Activation Maximization.}
We build on top of the Activation Maximization technique elaborated by Nguyen et. al \cite{nguyen2016synthesizing} (DGN-AM). DGN-AM utilizes the property of locality of codes:% with respected to generated images in Generator networks. Which means G
Generator codes which are close to each other would create similar looking images. DGN-AM was initially built to visualise the concept learnt by CNNs, by finding the code which maximised the activation of a particular output class in the classifier network. %This approach was initially aimed at visualizing the learnt knowledge of a convolutional neural network classifier. This is done by maximizing the activation of a particular output class probability neuron in a trained Classifier network, by feeding it images generated by a generator network. 
The maximization is achieved by doing gradient descent on the input generator codes with respect to the classifier neuronal activation, keeping everything else locked. The result is a synthetic image that has a high activation for a pre-determined output neuron.
%\textbf{[Please modify the text below (please find in comment something taken from another paper ** PLEASE CHANGE ***)]}
We modify this method by starting the maximization method from a code $K$ which corresponds to the a-priori input image, for example, an ugly urban image $I_i$. So for a given image $I_i$ which belongs to class $y_i$ (which could be the beauty neuron or the ugly neuron of the classifier $C$), the DGN-AM algorithm would perform Stochastic gradient descent on the generator codes of the a-priori image $I_i$ so as to maximize the target neuron $y_j$ (which could be beauty of ugly neuron of $C$ resulting in a synthetic image $\hat{I}_j$ generated by the generator $G$ from the code $ \hat{K} $. The whole optimization can be expressed as Equation \ref{eq:dgn-am}.
 \begin{equation}
  \hat{I_j}=G( \hat{K} ) : \underset{\hat{K}}{\arg\max}(C_j(G(\hat{K}))-\lambda||\hat{K}||)
  \label{eq:dgn-am}
 \end{equation}
Here $C_j$ corresponds to the activation of the neuron $j$ of the classifier $C$ , and $G$ is the generator network. $\lambda$ is the $L_2$ regularization factor.
The resulting output image $\hat{I_j}$ is a natural-like image, which maximizes the beauty neuron for our classifier. We hypothesize that because the process begins from an a-priori image, the resulting image is closest possible template to the ugly input image, but with the beauty neural activation maximized. Figure \ref{fig:BeautyExample} shows the activation maximization output in the center.
%***HERE****
%Given an input image $i$, \emph{DGN-AM} iteratively re-calculates the color of $i$'s pixels in  a way  the output image $\hat{i}_h$  both maximizes  the  activation of neuron $h$ and looks ``photo realistic'',  which is done by conditioning the maximization to an ``image prior". This is equivalent to finding the feature vector $f$ that maximizes the following expression:
% \begin{equation}
%  \hat{i}_h=G( f ) : \underset{f}{\arg\max}(\Phi_h(G(f))-\lambda||f||),
% \end{equation}
% where:
% \begin{itemize}
% \item $G(f)$ is the image synthetically generated from the candidate feature vector $f$;
% \item $\Phi_h(G(f))$ is the activation value of neuron $h$ in the scene classifier $\Phi_h$ (the value to be maximized);
% \item $\lambda$ is a $L_2$ regularization term.
% \end{itemize}
% Here the initialization of $f$ is key. If $f$ were to be initialized with random noise, then the resulting $G(f)$ would be the average representation of category $h$ (of, e.g., a garden). Instead, since $f$ is initialized with $i$, then the resulting $G(f)$ is $i$'s morphed version. That is, the details that make $i$ distinguishable from other images of category $h$  are removed (e.g., the details that make $i$ distinguishable from images of gardens are removed). Overall, the result of the iterations is the image $G(f)$ whose look is close to the average representation of category $h$. 

\begin{figure*}[h]
	\centering
	\includegraphics[width=0.5\linewidth]{Plot/GanCompare.png}
	\caption{Comparison of using the Default ImageNet GAN against Custom trained GAN for Activation maximization. By re-training the GAN on the test dataset, we can see improvement in terms of structure and colours in the generated images}
	\label{fig:GanComparison}
\end{figure*}

\subsection*{Step 3 Retrieving Images }
%In mathematical terms, we want to choose a target image $I'$ from $X$ so as to minimize $E(I' , \hat{I_j} )$ , where $E(I_1, I_2)$ is some error measure that quantifies visual error between two images. This image $I'$ is effectively a natural transformed image.
In this final step we find a target image $I'$ from the dataset that is closely aligned, in terms of some visual similarity metric $E(I_1, I_2)$, with the generated template image  $\hat{I_j}$ . The result of this exercise is to find the most similar looking image to an input image $I$ that maximizes a particular annotation class $y_j$.
%The visual differences in these two natural images can act as the subject of reasoning for the explain-ability.
The problem of finding images which are visually similar can be solved using image embeddings in a $N$ dimensional space $R^N$
We use a pre-trained deep  network, which is trained to classify scene types to a very high accuracy \cite{zhou2014learning} to extract the image embeddings. We extract a 4096 dimensional feature vector from the FC7 layer of the network to describe the the template image. We then extract feature vectors from the complete test dataset using the same process. We can now use the $L_2$ Norm to find pairwise distances in the $R^{4096}$. Formally with $N$ test natural images and a template image $\hat{i}$ we extract $v_{\hat{i}} \in R^{4096}$ and and find pairwise distances  $\{d_j \text{  }\forall j \in N\} \text{ where } d_j = L_2(v_j , v_{\hat{i}})$ 
We then find the target image by finding the $min(\{d_j\})$. For the sake of redundancy, we find the top 5 such matches for every template $\hat{i}$ generated from every ugly image $i$. These target images are what we call the transformed images.

\begin{figure*}[h]
	\centering
	\includegraphics[width=0.5\linewidth]{Plot/Example.png}
	\caption{Example of Beautification Process}
	\label{fig:BeautyExample}
\end{figure*}


\section{Evaluation}
\label{sec:evaluation}

The goal of our framework is to transform existing urban scenes into versions that: \emph{i)} people perceive more beautiful; \emph{ii)} contain urban elements typical of great urban spaces; \emph{iii)} are easy to interpret; and \emph{iv)} architects and urban planners find useful. To ascertain whether the framework meets that goal, next, we answer the following questions: 

\begin{description}
\item{\textbf{Q1}} Do individuals perceive our framework's scenes to actually be beautified?

\item{\textbf{Q2}}  Does our framework produce scenes that possess urban elements typical of great spaces?

\item{\textbf{Q3}}  Which urban elements are mostly associated with beautiful scenes?

\item{\textbf{Q4}}  Do architects and urban planners find our framework useful?

\end{description}


\subsection*{Q1 People's perceptions of beautified scenes}
To ascertain whether our framework's transformations are perceived by individuals as they are supposed to, we run a crowd-sourcing experiment on Amazon Mechanical Turk.  We randomly select 200 scenes, 100 beautiful and 100 ugly  based on whether they are at the bottom 10 and top 10 percentiles of the Trueskill's score distribution (Figure~\ref{fig:Trueskill}). Our framework then transforms each ugly scene into its beautified version, and each beautiful scene into its `uglified'. These scenes are arranged into pairs, each of which contains a beautiful scene and an ugly one. On  Mechanical Turk, we only select verified masters for our crowd-sourcing workers (those with an approval rate above 90\% during the past 30 days), pay them \$0.1 per  task,  and ask each of them to choose the beautiful scene in a pair.  We make sure to have at least 3 votes for each scene pair. Overall, our workers select the scenes that are actually beautiful 77.5\% of the times, suggesting that our framework's transformations are actually perceived by people to effectively be more beautiful. 


\subsection*{Q2 Are beautified scenes great urban spaces?}
To answer that question, we need to understand what makes a space great. After a careful review of the urban planning literature, we identify four factors~\cite{ewing2013measuring,alexander1977pattern} (summarized in Table~\ref{tab:Design_metrics}): great places mainly tend to be walkable, offer greenery, feel cozy, and be visually rich. 


\begin{table*}[h]
	\centering
	
	\resizebox{\linewidth}{!}{%
		\begin{tabular}{|c|p{14cm}|}
			\hline
			\textbf{Metric} & \textbf{Description}\\
			\hline
			Walkability  & Walkable streets are rated high on an aesthetic scale \cite{ewing2013measuring}. Walkable streets increase the social capital of a place and appeal to the exploring nature of human psyche. This implies that the urban space needs to address the fundamental need of people to walk and explore. This also implies that a walkable street must also be perceived as safe.\\
			\hline
			Green Spaces & Presence of Greenery is always pleasing to the eye. The literature always links urban beauty  to curated and well maintained green spaces, where social interactions can happen \cite{alexander1977pattern}. %, to be elements that bring a place 
			%together. 
			This 'social' aspect of greenery implies that dense forests or unkempt greens are not always related to the sense of beauty in urban scenes. \\
			\hline		
			Landmarks & Loosing a bearing in the city is not a very pleasant experience. Hence presence of recognisable and  guiding landmarks influences the perception of an urban space \cite{ewing2013measuring}.\\
			\hline
			Privacy-Openness &  A sense of privacy and a complimentary sense of openness are both influential in our perception of a place\cite{ewing2013measuring}. These values also tend to be related in an inverse 'U' fashion with beauty. \\ 
			\hline
			Visual Complexity & Visual complexity is a measure of how diverse a urban scene is. It manifests in terms of various design materials, textures and objects. Generally, visual complexity has an inverse 
			'U' relation with aesthetic values. The beauty and aesthetics of a place increases until it starts dropping because of 'too much' complexity\cite{ewing2013measuring}. \\
			\hline
	\end{tabular}}
	\caption{Urban Design Metrics}
	\label{tab:Design_metrics}
	%        \vspace{-5mm}
\end{table*}


To automatically extract visual cues related to these four factors, we select 500 ugly scenes and 500 beautiful ones at random, transform them into their opposite aesthetic qualities (i.e., ugly ones are beautified, and beautiful ones are `uglified'), and compare which urban elements related to the four factors distinguish uglified scenes from beautified ones. 

We extract labels from each of our 1,000 scenes using two computer vision algorithms. First, using PlacesNet~\cite{zhou2014learning}, we label our scenes according to a classification containing 205 labels (reflecting, for example, landmarks, natural elements), and retain the five labels with highest confidence scores for the scene. Second, using Segnet~\cite{badrinarayanan2015segnet}, we  label our scenes according to a classification containing 12 labels. Segnet is trained on dash-cam images, and the resulting lables are road, sky, trees,  buildings, poles, signage, pedestrians, vehicles, bicycles, pavement, fences, and road markings. 




\begin{figure}[h]
	\centering
	\includegraphics[width=\columnwidth]{Plot/taxonomyCount.png}
	\caption{Number of labels in specific urban design categories (on the $x$-axis) found in beautified scenes as opposed to those found in uglified scenes.}
	\label{fig:taxonomyCount}
\end{figure}


\begin{figure}[h]
	\centering
	\includegraphics[width=\columnwidth]{Plot/walkable_taxonomy.png}
	\caption{Count of specific walkability-related labels  (on the $x$-axis) found in beautified scenes minus the count of the same labels found in uglified scenes.}
	\label{fig:WalkableTnomy}
\end{figure}


%**************************************************
\noindent
\emph{H1 Beautified scenes tend to be walkable.}
We manually select only the labels that are related to walkability. These labels include, for example, \textit{abbey , plaza , courtyard, garden, picnic area, park}. To test hypothesis \emph{H1}, we count the number of walkability-related labels found in beautified scenes as opposed to those found in uglified scenes (Figure~\ref{fig:taxonomyCount}): the former contain twice as much walkability labels that the latter. We then consider the prevalence of specific labels Figure~\ref{fig:WalkableTnomy}, and  find that beautified scenes tend to show gardens, yards, small path. By contrast, uglified ones tend to show built environment features such as shop fronts and broad sidewalks. 


\mbox{ } \\
%**************************************************
\noindent
\emph{H2 Beautified scenes tend to offer green spaces.}
We manually select only the PlacesNet's labels that are related to greenery. These labels include, for example, \textit{fields, pasture , forest, ocean, beach}. Then, in our 1,000 scenes, to test hypothesis \emph{H2}, we count the number of nature-related labels found in beautified scenes as opposed to those found in uglified scenes (Figure~\ref{fig:taxonomyCount}): the former contain more than twice as much nature-related labels that the latter.  To test this hypothesis further, we compute the fraction of `tree' pixels (one of SegNet's label) in beautified and uglified scenes. We find that beautification adds  32\% of greenery pixels, and uglification removes 17\% of greenery pixels. 


%We classify the 205 scene labels of PlacesNet into 4 categories, \textbf{L}andmarks , \textbf{A}rchitectural , \textbf{W}alkable , \textbf{N}atural. Each category is inspired from  urban design literature \cite{ewing2013measuring}.  Labels like \textit{Abbey , Plaza , Courtyard, Garden, Picnic Area, Park , etc} fall into the category of \textit{Walkable}, where as labels like \textit{Mansion, Castle, Dam , Airport, etc} fall in the category of \textit{Landmarks}. Labels like \textit{Residential neighborhood, Motel, hotel, restaurant, etc} fall in the category of \textit{Architectural} and labels like {fields, pasture , forest, ocean, beach etc } fall in the category \textit{Natural}. %All in all  For an image, we then measure its Wakability according to how many of the top-5 labels fall in category W. Similarly, we quantify presence of Greenery and Landmarks according to the frequency of N and L labels.%These higher-level LAWN labels represent  broader urban design motifs that constitute a liveable city we can detect through PlaceNEt. 


\begin{figure*}[!t]
	\centering
	\hspace*{-5mm}
	\subfloat[]{
		\includegraphics[width=0.45\textwidth, height = 5cm ]{Plot/BinnedPlot.png}
		\label{fig:skyBinned}
	}
	\subfloat[]{
		\includegraphics[width=0.45\linewidth, height = 5cm ]{Plot/binnedPlot_complexity.png}
		\label{fig:complexity}
	}
\vspace{-0.4cm}
\label{fig:bin_figures}
\caption{The percentage of scenes ($y$-axis): (a) having an increasing presence of sky (on the $x$-axis); and (b) having an increasing level of visual richness  (on the $x$-axis).}
\vspace{-0.4cm}
\end{figure*}



\mbox{ } \\
%**************************************************
\noindent
\emph{H3 Beautified scenes tend to feel private and `cozy'.}
To  test hypothesis \emph{H3}, we count the fraction of pixels that Segnet labeled  as `sky' and show the results in a bin plot in Figure~\ref{fig:skyBinned}:  the $x$-axis has six bins each of which represents a given range of sky fraction, and the $y$-axis shows the percentage of beautified \emph{vs.} uglified scenes that fall into the corresponding bin.  Beautified scenes tend to be cozier (lower sky presence) than the corresponding original scenes.


\mbox{ } \\
%**************************************************
\noindent
\emph{H4 Beautified scenes tend to be visually rich.}
To quantify to which extent scenes are visually rich, we measure their visual complexity~\cite{ewing2013measuring} as  the amount of disorder in terms of distribution of (Segnet) urban elements in the scene: 
\begin{equation}
H(X) = -\sum p(i)\log p(i)
\label{eq:entropy} 
\end{equation}
where $i$ is the $i^{th}$ Segnet's label. The total number of labels is twelve. The higher $H(X)$, the  higher the scene's entropy, that is, the higher the scene's complexity. To test hypothesis \emph{H4}, we show the percentage of scenes that fall into a given bin of complexity scores (Figure~\ref{fig:complexity}): beautified images tend have low to medium complexity, while uglified scenes are of high complexity.


%After detecting LAWN categories through PlacesNet labels,  we compute, for each image, the difference between the category frequency before and after transformation (e.g. how many 'Walkable' labels are added after beautification?). We then plot the aggregated difference-distributions for beautified and uglified image sets in Fig \ref{fig:taxonomyCount}.% Because the transformation is directly dependent on maximizing the classifier's certainty about an image being beautiful or ugly, this method directly gives us a  beautiful or ugly urban scene.
%This method provides insights regarding how transformations change the presence of scene types.


%From the literature, it is conjectured that privacy is great when one
%looks at personal spaces, but when it comes to public settings, there is an inverse 'U' relation with how private a place feels like. 
%Too much privacy discourages the fundamental human urge to explore a mystery. Too much openness  alerts the primal urge to feel safe. 
%\par
%\textit{[H3] Sense of Privacy has an inverse 'U' relationship with the sense of beauty }
%\par
%What \textit{H3} suggests is that sense of privacy is not always associated with beauty. 



%\begin{figure}[h]
%	\centering
%	\includegraphics[width=\columnwidth]{Plot/BinnedPlot.png}
%	\caption{Binned Plot for Sky pixels across transformed images}
%	\label{fig:greenBinned}
%\end{figure}

%This effect can be individually seen from Fig \ref{fig:BuildingsCoverage} and Fig \ref{fig:TreeCoverage}. Beautification always prefers reduction of 
%visible buildings and increase in an overall green coverage \footnote{All distributions are tested using student-t tests and are found to have significantly hight T-statistic score (>>20) and p value << $10^{-5}$}. However this does not necessarily untangle the trade-off relationship of privacy-openness and beauty.

%To understand relation between openness and privacy, we repeat analysis as described in Section \ref{sec:vehicles}, for the amount of Sky and Tree pixels in a scene. The assumption here is these two objects in the scene are predominately driving the sense of openness. Adapting the Equation \ref{eq:regression} for the tree and sky pixel ratios we get different values for $\beta_1 , \beta_2 and \beta_3$. In this case the regression yields $\alpha = 0$ , $\beta_1 = 0.06$ , $\beta_2 = 0.10$ and $\beta_3 = -0.04$ . These values suggest that trees and sky, on their own have a positive impact on the beauty of a picture. 


%**************************************************
\subsection*{Q3 Urban elements of beautified scenes}

\begin{table}[t!]
	\centering
	\resizebox{\linewidth}{!}{%
	\begin{tabular}{|c|c|c|c|c|}
		\hline
		\textbf{Pair of urban elements} & \textbf{$\beta_1$}  & \textbf{$\beta_2$} & \textbf{$\beta_3$}  & Error Rate (Percentage)\\
		\hline
               \hline
		Buildings - Trees & -0.032 & 0.084  & 0.005  & 12.7 \\
		\hline
		Sky - Buildings & -0.08 & -0.11 & 0.064 & 14.4 \\
		\hline
		Roads - Vehicles  & -0.015  & -0.05 & 0.023  & 40.6 \\
		\hline
		Sky - Trees & 0.03 & 0.11 & -0.012 & 12.8  \\
		\hline
		Roads - Trees & 0.04  & 0.10 &  -0.031  & 13.5  \\
		\hline
		Roads - Buildings & -0.05  & -0.097  &  0.04  & 20.2  \\
		\hline
	\end{tabular}
	}
	\caption{Regression coefficients on a logistic run on a pair of predictors at the time.}
	\label{tab:regressioncoef}
    \vspace{-10mm}
\end{table}


To determine which urban elements are the best predictors of urban beauty and the extent to which they are so, we run a logistic regression, and, to ease interpretation, we do so on a pair of predictors at the time: 
\begin{equation}
Pr(beautiful) = logit^{-1}(\alpha + \beta_1 * V_1 + \beta_2 * V_2  + \beta_3 * V_{1}.V_{2} )
\label{eq:regression} 
\end{equation}
where $V1$ is the fraction of the scene's pixels marked with one Segnet's label, say, buildings (over the total number of pixels),  and $V2$ is the fraction of the scene's pixels marked with another label, say, trees. The result consists of three beta coefficients: $\beta_1$ reflects $V1$'s contribution in predicting beauty,  $\beta_2$ reflects $V2$'s contribution, and $\beta_3$ is the interaction effect, that, reflects the contribution of the dependency of $V1$ and $V2$ in predicting beauty. We run logistic regression on the five factors that have been found to be most predictive of urban beauty~\cite{quercia2014aesthetic, ewing2013measuring, alexander1977pattern}. 


Since we are using a logistic regression, the quantitative interpretation of the beta coefficients is eased by the ``divide by 4 rule''~\cite{vaughn2008data}: we can take $\beta$ coefficients and ``divide them by 4 to get an upper bound of the predictive difference corresponding to a unit difference'' in beauty~\cite{vaughn2008data}. For example, take the results in the first row of Table~\ref{tab:regressioncoef}. In the model $Pr(beautiful) = logit^{-1}(\alpha - 0.032 \cdot buildings + 0.084 \cdot trees + 0.005 \cdot  buildings \cdot trees)$, we can divide - 0.032/4 to get -0.008: a difference of 1 in the fraction of pixels being buildings corresponds to no more than a 0.8\% negative difference in the probability of the scene being beautiful. In a similar way, a difference of 1 in the fraction of pixels being trees corresponds to no more than a 0.021\% positive difference in the probability of the scene being beautiful. By considering the remaining results in Table~\ref{tab:regressioncoef}, we find that, across all pairwise comparisons, trees is the most positive element associated with beauty, while roads and buildings are the most negative elements. Since these results go in the direction one would expect, one might conclude that the scenes beautified by our framework are in line with previous literature, adding further external validity to our work. 




%**************************************************
\subsection*{Q4 Do architects and urban planners find it useful?}

\begin{figure*}[!t]
	\centering
	\hspace*{-5mm}
	\subfloat[]{
		\includegraphics[width=0.3\textwidth ]{Plot/DecisionMaking.png}
		\label{fig:decision}
	}
	\subfloat[]{
		\includegraphics[width=0.3\linewidth ]{Plot/ParticipationUrbanPlanning.png}
		\label{fig:participation}
	}
	\subfloat[]{
		\includegraphics[width=0.3\linewidth ]{Plot/PromoteGreenCities.png}
		\label{fig:promotion}
	}
%	\vspace{-0.4cm}
	\caption{Urban expert polling on the extent to which an interactive map of ``FaceLifted'' scenes promotes: (a) decision making; (b) citizen participation in urban planning; and (c) promotion of green cities.}
	\label{fig:pies}
	\vspace{-0.4cm}
\end{figure*}

To ascertain whether practitioners find FaceLift potentially useful, we built an interactive map of the city of Boston in which, for  selected points, we showed pairs of urban scenes before/after beautification \textbf{[picture?]}. We then sent that map along with a survey to 20 experts around the world in the areas of architecture, urban planning, and data visualization.  The experts had to complete tasks in which they rated FaceLift based on how well it supports decision making, participatory urbanism, and promotion of green spaces among the general public. The results are show in Figure~\ref{fig:pies}: according to our experts, the tool can very probably supports decision making, probably support participatory urbanism, and definitely promote green spaces.  These results are  qualitatively supported by our experts' comments, which included: ``\textit{The maps reveal patterns that might not otherwise be apparent}'',  ``\textit{The tool helps focusing on parameters to identify beauty in the city while exploring it}'',  and ``\textit{The metrics are nice. It made me think more about beautiful places needing a combination of criteria, rather than a high score on one or two dimensions. It made me realize that these criteria are probably spatially correlated}''.







\section{Conclusion}
\label{sec:discussion}

FaceLift is a  framework that automatically beautifies urban scenes by combining recent approaches of Generative Adversarial Networks and Deep Convolutional Networks. To make it usable by practitioners, the framework is also able to explain which urban elements have been added/removed during the beautification process. 

There are still important limitations though. One is data bias. The framework is as good as its training data, and more work has to go into collecting reliable ground truth data on human perceptions. This data should ideally be stratified according to the people's characteristics that  impact their perceptions. The other main limitation is that generative models are hard to control, and more work has to go into offering principled ways of fine-tuning the generative process.

Despite these limitations, FaceLift has the potential to support urban interventions  in scalable  and replicable ways: it can be applied to an entire city (scalable), across a variety of cities (replicable). To turn existing spaces into something more beautiful, that will still be the duty of architecture. Yet, with technologies similar to FaceLift more readily available, the complex job of recreating restorative spaces in an increasingly urbanized world will be greatly simplified.  


After all, ``we delight in complexity to which genius have lent an appearance of simplicity.''~\cite{de2008architecture} In the context of future work, that genius is represented by future technologies that will help us deal with the complexity of our cities.



%\subsection{Limitations and biases}
%Like any supervised deep learning based framework, this work is only able to learn what is present in the data. Hence the method of acquiring annotations  for urban images can introduce huge biases in the model. The current model is trained on images acquired from the study on streetscore \cite{naik2014streetscore}. However their annotation is open to general public and there is not way we can remove biases that come with culture and location, in a highly subjective effect like beauty. Moreover because the pair wise choice is simply done by clicking one of the two images, the data might have noise introduced by non-serious participants. Such biases are bound to be picked up by the deep learning model. One can argue that the preference of our model for greenery , is a form of bias in the data. Another bias introduced because of data is the model's lack of preference to pedestrians. This bias was established well in advance because Google tries to remove most of the people from their street view images for privacy reasons. Hence people, which make up a major aspect of urban vitality, are completely missing from most dataset images and hence from the facelift transformations. 
%Another Limitation of our work is in the metric formation. The computational metrics developed to capture the real urban design metrics are designed using heuristics. There needs to be more crowd and expert validation to establish the validity of their formulation. 
%
%\subsection{Future work}
%The framework is generalizable for geotagged and annotated images. The aim of this paper is to propose a framework with uses state of art methods in generative models to understand perception of emotions in urban images and explain them. As an extension, understanding how intervention would look like against outcome variables such as depression, safety or mental well-being in general would be very valuable.

\balance
\bibliographystyle{ACMbib}
\bibliography{urbanEmotions} 


\end{document}